\chapter*{ВВЕДЕНИЕ}
\addcontentsline{toc}{chapter}{ВВЕДЕНИЕ}
Данная работа посвящена моделированию сцены, состоящей из нескольких пузырей, динамически образующих кластеры (в кластере не более двух пузырей), соединяющихся воедино и отталкивающихся. Результат взаимодействия соприкоснувшихся пузырьков рассчитывается с учётом угла соприкосновения, поверхностного натяжения и разницы давлений.

\textbf{Цель:} моделирование изображения трёхмерной сцены с пузырьковыми кластерами.

\textbf{Задачи:}
\begin{enumerate}[label={\arabic*)}]
	\item провести обзор существующих методов визуализации трёхмерной сцены, обосновать выбор метода;
	\item описать метод взаимодействия пузырей;
	\item реализовать возможность создания отдельных пузырей;
	\item реализовать возможность перемещения отдельных пузырей;
	\item реализовать взаимодействия пузырей при построении трёхмерного изображения сцены в динамике.
\end{enumerate}