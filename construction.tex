\chapter{Конструкторский раздел}
\section{Математические основы метода математического моделирования}
\subsection{Математические модели}
Математическая модель, выбранная для пузырьков, -- это сфера, которая в свою очередь состоит из двух полусфер и круговой поверхности. Данный выбор обусловлен упрощением геометрической формы пузырьков, что позволяет использовать более простые математические уравнения и алгоритмы для описания их поведения и взаимодействия.

Круговая поверхность -- это упрощённая до плоскости мембрана между пузырьками в кластере.

Параметрами математической модели пузырька являются:
\begin{itemize}
	\item координаты центра в трёхмерном пространстве;
	\item радиус.
\end{itemize}

Для каждой полусферы составляющей модель, характерны следующие параметры:
\begin{itemize}	
	\item координаты центра в трёхмерном пространстве (такие же как и у общей сферы);
	\item высота сегмента;
	\item направление (от центра сферы в две противоположные стороны).
\end{itemize}

Причиной разделения сферы на две полусферы является первый пункт раздела \verb|"|Модель взаимодействия двух пузырей при соприкосновении\verb|"|~\ref{sec:model_of_contact}. При образовании кластера обе сферы должны быть обрезаны плоскостью соприкосновения их поверхностей, что при модели состоящей из двух полусфер требует лишь уменьшения высоты соприкасающихся полусфер и редактирования направлений.

Математической моделью пузырькового кластера в свою очередь состоит из двух уже описанных выше сфер. 

\subsection{Основные математические уравнения, использованные для решения задачи}
Формулы~\ref{eq:distance}, \ref{eq:angle} заслуживают упоминания, хоть и являются достаточно простыми, поскольку лежат в основе любого взаимодействия описанных выше математических моделей.

\textbf{Формула нахождения расстояния между центрами сфер:}
\begin{equation} \label{eq:distance}
	d = \sqrt{(X_1 - X_2)^2 + (Y_1 - Y_2)^2 + (Z_1 - Z_2)^2}
\end{equation}
где: 
\begin{itemize}	
	\item d -- расстояние между центрами сфер;
	\item ($X_1$, $Y_1$, $Z_1$) -- координаты центра первой сферы;
	\item ($X_2$, $Y_2$, $Z_2$) -- координаты центра первой сферы.
\end{itemize}

\textbf{Формула нахождения угла соприкосновения:}
\begin{equation} \label{eq:angle}
	\theta = \arccos\left(\frac{AB \cdot AC}{|AB| \cdot |AC|}\right) \cdot \frac{180}{\pi}
\end{equation}
где:
\begin{itemize}	
	\item $AB$ -- вектор от центра первой сферы до точки пересечения;
	\item $AC$ -- вектор от центра второй сферы до точки пересечения;
	\item $P$ -- точка пересечения;
	\item $|AB|$ -- длина вектора $AB$;
	\item $|AC|$ -- длина вектора $AC$.
\end{itemize}

\textbf{Формула нахождения радиуса сфера, чей объём равен сумме объёмов двух других сфер:}
\begin{equation}
	r_{\text{new}} = \left( \frac{3}{4\pi} \left( \frac{4}{3} \pi r_1^3 + \frac{4}{3} \pi r_2^3 \right) \right)^{\frac{1}{3}}
\end{equation}

где $r_1$ и $r_2$ — радиусы первой и второй сферы соответственно.

\section{Разработка алгоритма метода моделирования}

Схема алгоритма расстановки пузырьков представлена на рисунке~\ref{fig:bubble_positioning}, алгоритмы, который он в себя включает, а именно алгоритм слияния в один большой пузырь, отталкивание пузырьков друг от друга и создания пузырькового кластера, представлены на рисунках~\ref{fig:merge_bubbles}, \ref{fig:push_bubbles_apart}, \ref{fig:create_bubble_cluster} соответственно.

\begin{figure}[h]
	\centering
	\includegraphics[width=1\linewidth]{pictures/"Расстановка пузырьков.png"}
	\caption{Алгоритм расстановки пузырьков}
	\label{fig:bubble_positioning}
\end{figure}
\begin{figure}[h]
	\centering
	\includegraphics[width=1\linewidth]{pictures/"Слияние в один большой пузырь.png"}
	\caption{Алгоритм расстановки пузырьков, слияние в один большой пузырь }
	\label{fig:merge_bubbles}
\end{figure}
\begin{figure}[h]
	\centering
	\includegraphics[width=0.79\linewidth]{pictures/"Отталкивание пузырьков друг от друга.png"}
	\caption{Алгоритм расстановки пузырьков, отталкивание пузырьков друг от друга}
	\label{fig:push_bubbles_apart}
\end{figure}
\begin{figure}[h]
	\centering
	\includegraphics[width=1\linewidth]{pictures/"Создание пузырькового кластера.png"}
	\caption{Алгоритм расстановки пузырьков, создание пузырькового кластера}
	\label{fig:create_bubble_cluster}
\end{figure}
\clearpage
Схема алгоритма обратной трассировки лучей представлена на рисунках~\ref{fig:rendering} и~\ref{fig:trace_ray}. Важно отметить, что алгоритм обратной трассировки лучей для конкретного луча является рекурсивным.
\begin{figure}[h]
	\centering
	\includegraphics[width=0.745\linewidth]{pictures/"rendering.png"}
	\caption{Алгоритм обратной трассировки лучей}
	\label{fig:rendering}
\end{figure}
\begin{figure}[h]
	\centering
	\includegraphics[width=1\linewidth]{pictures/"trace_ray.png"}
	\caption{Алгоритм обратной трассировки лучей для одного луча}
	\label{fig:trace_ray}
\end{figure}
\clearpage
Поскольку в техническом задании есть следующие слова: "...моделированию сцены, состоящей из нескольких пузырей, динамически образующих кластеры...", а алгоритм обратной трассировки лучей, работает слишком долго для динамических преобразований, была введена ещё одна математическая модель -- окружность. Этот элемент является двумерным слепком перемещающейся сферы на экране.  Преобразование параметров сферы ($(X, Y, Z)$ -- координаты центра сферы, $R$ -- радиус сферы) в параметры окружности на экране ($(x, y)$ -- координаты центра окружности, $r$ -- радиус сферы) и обратно, представлено следующими выражениями~\ref{eq:sphere_to_circle} и~\ref{eq:circle_to_sphere}.
\begin{equation} \label{eq:sphere_to_circle}
\begin{gathered}
	scaleFactor = W / 1.0 \\
	r = R \cdot \left( \frac{1}{d} \right) \\
	x = \left( \frac{W}{2} + X \cdot \left( \frac{1}{d} \right) \cdot scaleFactor \right) \\
	y = \left( \frac{H}{2} - Y \cdot \left( \frac{1}{d} \right) \cdot scaleFactor \right) \\
\end{gathered}
\end{equation}
где: 
\begin{itemize}	
	\item $W$ -- ширина экрана;
	\item $H$ -- высота экрана, ($W = H$);
	\item $d$ -- расстояние по оси $z$ от позиции камеры до центра сферы.
\end{itemize}


\begin{equation} \label{eq:circle_to_sphere}
\begin{gathered}
	scaleFactor = W / 1.0 \\
	X = \frac{(x + shift_x) - \frac{W}{2}}{scaleFactor} \cdot \left( \frac{1.0}{d} \right) \\
	Y = \frac{\frac{H}{2} - (y + shift_y)}{scaleFactor} \cdot \left( \frac{1.0}{d} \right)
\end{gathered}
\end{equation}
где: 
\begin{itemize}	
	\item $W$ -- ширина экрана;
	\item $H$ -- высота экрана, ($W = H$);
	\item $d$ -- расстояние по оси $z$ от позиции камеры до центра сферы;
	\item $shift_x$ -- перемещение относительно оси $ox$;
	\item $shift_y$ -- перемещение относительно оси $oy$.
\end{itemize}